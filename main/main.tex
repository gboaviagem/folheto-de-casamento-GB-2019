\documentclass[a5paper,9pt]{memoir}
% A classe memoir aceita o tamanho de fonte 9pt

\usepackage[utf8]{inputenc} % coding
\usepackage[portuguese]{babel}
\usepackage{graphicx}
\usepackage{amsmath,amsfonts,amssymb,amsthm}
\usepackage{tabulary}
\usepackage{longtable}
\usepackage{booktabs}
\usepackage{array}% for extended column definitions    
\interdisplaylinepenalty=2500 % To restore IEEEtran’s ability to automatically do page breaks within multiline equations
\usepackage[width=0.00cm, left=2.50cm, right=2.50cm, top=2.00cm, bottom=2.00cm]{geometry} % configura as margens
\usepackage{url}	
\usepackage{hyperref}
\usepackage{subfig}
\usepackage{xcolor}
\usepackage{url} % Permite exibição de sites de forma organizada
\usepackage{nicefrac}
\usepackage{cite}
\usepackage{lipsum}
\usepackage{booktabs}
\usepackage{lipsum}
\usepackage{blindtext}
\usepackage{bbding}

\usepackage{paracol}
\usepackage{ragged2e}
\usepackage{multicol}

\let\footruleskip\undefined
\usepackage{fancyhdr}
\pagestyle{fancy}
\fancyhf{}
\renewcommand{\headrulewidth}{0pt}
\cfoot{\thepage}

% FONTE
\usepackage[T1]{fontenc}
\usepackage{librebaskerville}

\definecolor{mygray}{gray}{0.4}
\newcommand{\red}{\textcolor{red}} 
\newcommand{\blue}{\textcolor{blue}}
\newcommand{\BG}{\textbf{Bár\-ba\-ra} e \textbf{Gui\-lher\-me}}
\newcommand{\GB}{\textbf{Guilherme} e \textbf{B\'arbara}}

\setlength\parindent{0pt}
\setlength{\parskip}{\baselineskip}%
\renewcommand{\baselinestretch}{1.3}


\begin{document}

\thispagestyle{empty}

\noindent ``No Novo Testamento, com a Nova Aliança inaugurada por Jesus no Calv\'a\-rio, a preço do seu Sangue, o casamento terreno passaria a ser sinal de um pacto ainda mais forte e firme, pois selado com o sacrifício do Salvador que era, Ele mesmo, Deus: o casamento é sinal da Aliança entre Jesus e a Igreja, e esta Aliança é um verdadeiro casamento místico. A Cruz é uma boda! E a Missa, atualizando a Cruz, tornando-a presente, é, então, um casamento também!''
%É por isso que a Igreja incentiva a que os casamentos católicos sejam realizados na Missa, para que a união entre os noivos seja selada com o tornar presente da união entre a Igreja e o Senhor Jesus Cristo.''

\vspace{0.5cm}
\hfill --- \emph{``Família católica, Igreja doméstica'', Rafael e Aline Brodbeck.}

\pagebreak

{\color{white}Blank}
\thispagestyle{empty}
\vfill
\begin{center}
\begin{minipage}{0.76\linewidth}
\centering
\emph{As letras dos cantos em latim e ingl\^es encontram-se traduzidas ao fim deste folheto.}
\end{minipage}
\end{center}

\pagebreak

\begin{center}
%\bfseries
SANTA MISSA \emph{PRO SPONSIS} \\ {\large COM CELEBRAÇÃO DO MATRIMÔNIO}
\end{center}
%\vspace{0.3cm}

\begin{center}
\bfseries
RITOS INICIAIS
\end{center}

\textbf{P.} Em nome do Pai, e do Filho, e do Espírito Santo.\\
\textbf{R.} Amém.

\textbf{P.} A graça de nosso Senhor Jesus Cristo, o amor do Pai e a comunhão do Espírito Santo estejam convosco!\\
\textbf{R.} Bendito seja Deus, que nos reuniu no amor de Cristo!

\textbf{P.} \BG, a Igreja participa da vossa alegria e vos recebe de cora\c c\~ao, assim como a vossos pais, parentes e amigos, neste dia em que, diante de Deus, nosso Pai, ireis firmar entre v\'os uma alian\c ca por toda a vida. Que o Senhor vos ou\c ca neste dia de tanta felicidade, e vos mande o aux\'ilio celeste, e assim vos conserve por muito tempo, e vos conceda muitas gra\c cas, segundo o vosso cora\c c\~ao, e vos realize todas as vossas aspira\c c\~oes.

{\itshape \color{mygray}Omite-se o Ato Penitencial.}
{
\begin{center}
ORAÇÃO DA COLETA
\end{center}

{\itshape \color{mygray}``O Sacerdote convida o povo a rezar, todos se conservam em silêncio com o sacerdote por alguns instantes, tomando consciência de que estão na presença de Deus e formulando interiormente os seus pedidos. Depois o sacerdote diz a oração que costumar chamar de coleta, pela qual exprime a índole da celebração'' (IGMR, n.54).}

\textbf{P.} Oremos. Concedei, \'o Deus, a \BG, que v\~ao unir-se pelo sacramento do matrim\^onio, progredir na f\'e que professam e enriquecer de filhos vossa Igreja. Por nosso Senhor Jesus Cristo, vosso Filho, na unidade do Esp\'irito Santo.\\
\textbf{R.} Amém.


\begin{center}
\bfseries
LITURGIA DA PALAVRA
\end{center}

{\itshape \color{mygray}\textbf{I LEITURA} \hfill Ap 19, 1.5-9a}
\vspace{-0.3cm}
\begin{center}
\itshape \color{mygray} \small
\guillemotleft Felizes os convidados para o banquete das núpcias do Cordeiro\guillemotright
\end{center}
\vspace{-0.3cm}

Leitura do Livro do Apocalipse de S\~ao Jo\~ao

\columnratio{0.15}
\begin{paracol}{2}
	{\color{white}Blank.}
	
	\switchcolumn
	
	\itshape
	Eu, Jo\~ao, ouvi um forte rumor de uma grande multid\~ao no c\'eu, que clamava: ``Aleluia! A salva\c c\~ao, a gl\'oria e o poder pertencem ao nosso Deus''. Ent\~ao uma voz saiu do trono, convidando: ``Louvai o nosso Deus, todos os seus servos e todos v\'os que o temeis, pequenos e grandes''. Ouvi tamb\'em o rumor de uma grande multid\~ao. Parecia o fragor de \'aguas torrenciais e o ribombar de fortes trov\~oes. A multid\~ao clamava: ``Aleluia! O Senhor, nosso Deus, o Todo poderoso passou a reinar. Fiquemos alegres e contentes, e demos gl\'oria a Deus, porque chegou o tempo das n\'upcias do Cordeiro. Sua esposa j\'a se preparou. Foi-lhe dado vertir-se com linho brilhante e puro''. (O linho significa as obras justas dos santos). E um anjo me disse: ``Escreve: felizes s\~ao os convidados para o banquete das n\'upcias do Cordeiro''.
%	Eu, João, ouvi como que a voz poderosa de uma grande multidão, que dizia no Céu: \guillemotleft Aleluia! A salvação, a glória e o poder pertencem ao nosso Deus!\guillemotright E do trono saiu uma voz que dizia: Louvai o nosso Deus, vós todos os seus servos, vós que O temeis, pequenos e grandes!\guillemotright. Depois ouvi como que a voz de uma grande multidão, como o marulhar de águas caudalosas, como o ribombar de fortes trovões, aclamando: \guillemotleft Aleluia, porque reina o Senhor, nosso Deus onipotente. Alegremo-nos e exultemos e demos-Lhe glória, porque chegou o tempo das núpcias do Cordeiro e a sua Esposa está preparada: foi-lhe concedido que vestisse linho fino e resplandecente \guillemotright. Esse linho são as obras justas dos santos. Disse o Anjo: \guillemotleft Escreve: `felizes os convidados para o banquete das núpcias do Cordeiro'\guillemotright.
\end{paracol}

Palavra do Senhor.\\
\textbf{R.} Graças a Deus.

\vspace{-0.8cm}
\begin{center}
\begin{minipage}[t]{0.5\linewidth}
\itshape \color{mygray}\textbf{SALMO\\ RESPONSORIAL}
\end{minipage}~
\begin{minipage}[t]{0.5\linewidth}
\itshape \color{mygray}
\hfill Sl 127 (128), 1-2.3.4-5
\end{minipage}
\end{center}

{\itshape \color{mygray}Refr\~ao:} \textbf{Felizes os que temem o Senhor.}

Feliz \'es tu se temes o Senhor\\
e trilhas seus caminhos!\\
Do trabalho de tuas m\~aos h\'as de viver,\\
ser\'as feliz, tudo ir\'a bem! {\itshape \color{mygray}R.}

A tua esposa \'e uma videira bem fecunda\\
no cora\c c\~ao de tua casa;\\
os teus filhos s\~ao rebentos de oliveira\\
ao redor de tua mesa. {\itshape \color{mygray}R.}

Ser\'a assim aben\c coado todo homem\\
que teme o Senhor.\\
O Senhor te aben\c coe de Si\~ao,\\
cada dia de tua vida. {\itshape \color{mygray}R.}

\vspace{0.3cm}
{\itshape \color{mygray}\textbf{II LEITURA} \hfill 
Ef 5, 2a.21-33}
\vspace{-0.3cm}
\begin{center}
	\itshape \color{mygray} \small
	\guillemotleft Este mist\'erio \'e grande, e eu o interpreto em relação a Cristo e à Igreja\guillemotright
\end{center}
\vspace{-0.3cm}

Leitura da Carta de S\~ao Paulo aos Ef\'esios

\columnratio{0.15}
\begin{paracol}{2}
	{\color{white}Blank.}
	
	\switchcolumn
	
	\itshape
	Irmãos:\\
	Vivei no amor, como Cristo nos amou e se entregou a si mesmo a Deus por n\'os. V\'os que temeis a Cristo, sede sol\'icitos uns para com os outros. As mulheres sejam submissas aos seus maridos como ao Senhor. Pois o marido \'e a cabe\c ca da mulher, do mesmo modo que Cristo \'e a cabe\c ca da Igreja, ele, o salvador do Corpo. E como a Igreja \'e sol\'icita por Cristo, sejam as mulheres sol\'icitas em tudo pelos seus maridos. Maridos, amai as vossas mulheres como o Cristo amou a Igreja e se entregou por ela. Ele quis assim torn\'a-la santa, purificando-a com o banho da \'agua unida \`a Palavra. Ele quis apresent\'a-la a si mesmo espl\^endida, sem mancha nem ruga, nem defeito algum, mas santa e irrepreens\'ivel. Assim \'e que o marido deve amar a sua mulher, como ao seu pr\'oprio corpo. Aquele que ama a sua mulher ama-se a si mesmo. Ningu\'em jamais odiou a sua pr\'opria carne. Ao contr\'ario, alimenta-a e cerca-a de cuidados, como o Cristo faz com a sua Igreja; e n\'os somos membros do seu corpo! Por isso, o homem deixar\'a seu pai e sua m\~ae e se unir\'a \`a sua mulher, e os dois ser\~ao uma s\'o carne. Este mist\'erio e grande, e eu o interpreto em rela\c c\~ao a Cristo e \`a Igreja. Em todo o caso, cada um, no que lhe toca, deve amar a sua mulher como a si mesmo; e a mulher deve respeitar o seu marido.
%	Caminhai na caridade, a exemplo de Cristo, que nos a\-mou e se entregou por nós. Sede submissos uns aos outros no temor de Cristo. As mulheres submetam-se aos maridos como ao Senhor, porque o marido é a cabeça da mulher, como Cristo é a cabeça da Igreja, seu Corpo, do qual é o salvador. Ora, como a Igreja se submete a Cristo, assim também as mulheres se devem submeter em tudo aos maridos. Maridos, amai as vossas mulheres, como Cristo amou a Igreja e se entregou por ela. Ele quis santificá-la, purificando-a no baptismo da água pela palavra da vida, para a apresentar a si mesmo como Igreja cheia de glória, sem mancha nem ruga, nem coisa alguma semelhante, mas santa e imaculada. Assim devem os maridos amar as suas mulheres, como os seus corpos. Quem ama a sua mulher ama-se a si mesmo. Ninguém, de fato, odiou jamais o seu corpo, antes o alimenta e lhe presta cuidados, como Cristo à Igreja; porque nós somos membros do seu Corpo. Por isso, o homem deixará pai e mãe, para se unir à sua mulher, e serão dois numa só carne. É grande este mistério, digo-o em relação a Cristo e à Igreja. Portanto, cada um de vós ame a sua mulher como a si mesmo e a mulher respeite o marido.
\end{paracol}

Palavra do Senhor.\\
\textbf{R.} Graças a Deus.

%\begin{center}
%ACLAMAÇÃO AO EVANGELHO
%\end{center}

\noindent {\color{mygray} \itshape Todos colocam-se de p\'e, para a aclama\c c\~ao ao Evangelho.}
%\columnratio{0.5}
%\begin{paracol}{2}
%	\small
%	\itshape
%	\definecolor{mygray}{gray}{0.4}
%	\color{mygray}
%	\setlength{\parindent}{0em}
%	
%Buscai primeiro o reino de Deus\\
%E a sua justiça\\
%E tudo mais o será acrescentado\\
%\textbf{Aleluia}! \textbf{Aleluia}!
%\switchcolumn
%Não só de pão o homem viverá\\
%Mas de toda palavra\\
%Que procede da boca de Deus\\
%\textbf{Aleluia}! \textbf{Aleluia}!
%\end{paracol}

% EVANGELHO
\vspace{0.3cm}
{\itshape \color{mygray}\textbf{EVANGELHO} \hfill 
Jo 2, 1-11}
\vspace{-0.3cm}
\begin{center}
	\itshape \color{mygray} \small
	\guillemotleft Este foi o in\'icio dos sinais de Jesus. Ele o realizou em Can\'a da Galileia.\guillemotright
\end{center}
\vspace{-0.3cm}

\textbf{P.} O Senhor esteja convosco.\\
\textbf{R.} Ele est\'a no meio de n\'os.

\textbf{P.} {\color{mygray}\CrossMaltese}  Proclama\c c\~ao do Evangelho de Jesus Cristo, segundo João.\\
\textbf{R.} Gl\'oria a v\'os, Senhor.

\columnratio{0.15}
\begin{paracol}{2}
	{\color{white}Blank.}
	
	\switchcolumn
	
	\itshape
	Naquele tempo,\\
	houve um casamento em Can\'a da Galileia. A m\~ae de Jesus estava presente. Tamb\'em Jesus e seus disc\'ipulos tinham sido convidados para o casamento. Como o vinho veio a faltar, a m\~ae de Jesus lhe disse: ``Eles n\~ao t\^em mais vinho''. Jesus respondeu-lhe: ``Mulher, por que dizes isso a mim? Minha hora ainda n\~ao chegou''. Sua m\~ae disse aos que estavam servindo: ``Fazei o que ele vos disser''. Estavam seis talhas de pedra colocadas a\'i para a purifica\c c\~ao que os judeus costumam fazer. Em cada uma delas cabiam mais ou menos cem litros. Jesus disse aos que estavam servindo: ``Enchei as talhas de \'agua''. Encheram-nas at\'e a boca. Jesus disse: ``Agora tirai e levai ao mestre-sala''. E eles levaram. O mestre-sala experimentou a \'agua, que se tinha transformado em vinho. Ele n\~ao sabia de onde vinha, mas os que estavam servindo sabiam, pois eram eles que tinham tirado a \'agua. O mestre-sala chamou ent\~ao o noivo e lhe disse: ``Todo o mundo serve primeiro o vinho melhor e, quando os convidados j\'a est\~ao embriagados, serve o vinho menos bom. Mas tu guardaste o vinho melhor at\'e agora!'' Este foi o in\'icio dos sinais de Jesus. Ele o realizou em Can\'a da Galileia e manifestou a sua gl\'oria, e seus disc\'ipulos creram nele.
%	realizou-se um casamento em Caná da Galileia e estava lá a Mãe de Jesus. Jesus e os seus discípulos foram também convidados para o casamento. A certa altura faltou o vinho. Então a Mãe de Jesus disse-Lhe: «Não têm vinho». Jesus respondeu-Lhe: «Mulher, que temos nós com isso? Ainda não chegou a minha hora». Sua Mãe disse aos serventes: «fazei tudo o que Ele vos disser». Havia ali seis talhas de pedra, destinadas à purificação dos judeus, e cada uma levava duas ou três medidas. Disse-lhes Jesus: «Enchei essas talhas de água». Eles en\-cheram-nas até acima. Depois disse-lhes: «Tirai agora e levai ao chefe de mesa». E eles levaram. Quando o chefe de mesa provou a água transformada em vinho, – ele não sabia de onde viera, pois só os serventes, que tinham tirado a água, sabiam – chamou o noivo e disse-lhe: «Toda a gente serve primeiro o vinho bom e, depois de os convidados terem bebido bem, serve o inferior. Mas tu guardaste o vinho bom até agora». foi assim que, em Caná da Galileia, Jesus deu início aos seus milagres. Manifestou a sua glória e os discípulos acreditaram n'Ele.
\end{paracol}
%\pagebreak
Palavra da Salva\c c\~ao.\\
\textbf{R.} Gl\'oria a v\'os, Senhor.

%\vspace{-0.5cm}
%\begin{center}
%HOMILIA
%\end{center}

\vspace{-0.2cm}
%\pagebreak
\begin{center}
HOMILIA \\
\vspace{0.2cm}
{\bfseries
RITO DO MATRIM\^ONIO}
\end{center}
\vspace{-0.2cm}

\noindent {\itshape\color{mygray} Todos colocam-se de p\'e.}

\textbf{P.} Caros noivos, \BG, viestes a esta igreja, para que, na presen\c ca do sacerdote e da comunidade crist\~a, a vossa decis\~ao de contrair Matrim\^onio seja marcada por Cristo com o sinal sagrado. Cristo aben\c coa com generosidade o vosso amor conjugal. J\'a vos tendo consagrado pelo batismo, vai enriquecer e fortalecer-vos agora com o sacramento do Matrim\^onio, para que sejais fieis um ao outro por toda a vida e possais assumir todos os deveres do Matrim\^onio.

\begin{center}
DI\'ALOGO ANTES DO CONSENTIMENTO
\end{center}

\textbf{P.} \BG, viestes aqui para unir-vos em Matrimônio. Por isso, eu vos pergunto perante a Igreja:\\\'E de livre e espont\^anea vontade que o fazeis?\\
\textbf{Noivos:} Sim!

\textbf{P.} Abra\c cando o Matrim\^onio, ides prometer amor e fidelidade um ao outro. \'E por toda a vida que o prometeis?\\
\textbf{Noivos:} Sim!

\textbf{P.} Estais dispostos a receber com amor os filhos que Deus vos confiar, educando-os na lei de Cristo e da Igreja?\\
\textbf{Noivos:} Sim!

\pagebreak
\begin{center}
CONSENTIMENTO
\end{center}

\textbf{P.} Para manifestar o vosso consentimento em selar a sagrada alian\c ca do Matrim\^onio, diante de Deus e da Igreja, aqui reunida, dai um ao outro a m\~ao direita.

{\itshape \color{mygray}Os noivos unem as mãos.}

\columnratio{0.15}
\begin{paracol}{2}
{\color{white}Blank.}

\switchcolumn

\itshape
\textbf{Noivo:} Eu, \textbf{Guilherme}, te recebo \textbf{B\'arbara}, por minha esposa e te prometo ser fiel, amar-te e respeitar-te na alegria e na tristeza, na sa\'ude e na doen\c ca, todos os dias da nossa vida.

\textbf{Noiva:} Eu, \textbf{B\'arbara}, te recebo \textbf{Guilherme}, por meu esposo e te prometo ser fiel, amar-te e respeitar-te na alegria e na tristeza, na sa\'ude e na doen\c ca, todos os dias da nossa vida.
\end{paracol}

\begin{center}
ACEITAÇÃO DO CONSENTIMENTO
\end{center}

\textbf{P.} Deus confirme este compromisso que manifestastes perante a Igreja e derrame sobre v\'os as suas b\^en\c c\~aos! Ningu\'em separe o que Deus uniu!\\
Bendigamos ao Senhor!\\
\textbf{R.} Graças a Deus!

%\begin{center}
%ENTRADA DAS ALIANÇAS
%\end{center}
%\definecolor{mygray}{gray}{0.4}
%\setlength{\parindent}{0em}
%{\color{mygray}
%\itshape
%\small
%\textbf{Amar-te mais} [trecho]\\
%(Davidson Silva)
%	
%Amar-te mais que a mim mesmo\\
%Amar-te mais que tudo que há aqui,\\
%Amar-te mais que aos mais queridos,\\
%Amar-te e dar a vida só por ti (bis)
%}

\begin{center}
ENTRADA DAS ALIANÇAS \\ \vspace{0.1cm}
BÊNÇÃO E ENTREGA DAS ALIANÇAS
\end{center}

\textbf{P.} Deus aben\c coe {\itshape \color{mygray}\CrossMaltese} estas alian\c cas que ides entregar um ao outro em sinal de amor e fidelidade.\\
\textbf{R.} Amém.

{\itshape \color{mygray}Os esposos colocam no dedo anelar do cônjuge, dizendo:}

\columnratio{0.15}
\begin{paracol}{2}
{\color{white}Blank.}

\switchcolumn

\itshape
{\itshape \color{mygray}N.}, recebe esta alian\c ca em sinal do meu amor e da minha fidelidade. Em nome do Pai, e do Filho, e do Espírito Santo.
\end{paracol}

%{\itshape \color{mygray}Neste momento toda a comunidade pode cantar um hino ou um cântico de louvor.}

\begin{center}
%\bfseries
ORAÇÃO UNIVERSAL
\end{center}

\textbf{P.} Irmãos e irmãs em Cristo,
acompanhemos com as nossas preces esta nova fam\'ilia, para que o amor m\'utuo destes esposos cres\c ca cada dia mais, e proteja Deus a todas as fam\'ilias do mundo.\\

\columnratio{0.15}
\begin{paracol}{2}
{\color{white}Blank.}
	
\switchcolumn

1. Por estes novos esposos {\textbf{B\'arbara} e \textbf{Guilherme}}, e pela felicidade de sua fam\'ilia, {para que a Virgem Auxiliadora intervenha sempre por eles com solicitude,} rezemos ao Senhor.

{\itshape \textbf{R.} Senhor, escutai a nossa prece.}

2. Por seus pais, parentes e amigos e por todos os que os ajudaram neste casamento, rezemos ao Senhor.

3. Pelos jovens que se preparam para o casamento, e por todos os que Deus chama a um outro estado de vida, rezemos ao Senhor.

4. Por todas as fam\'ilias do mundo e pela paz entre todos, rezemos ao Senhor.

5. Por todos os membros das nossas fam\'ilias que j\'a partiram deste mundo, e por todos os falecidos, rezemos ao Senhor.

6. Pela Igreja, Povo santo de Deus, e pela unidade de todos os crist\~aos, rezemos ao Senhor.

%7. Por \BG, para que a Virgem Auxiliadora intervenha por eles com solicitude, co\-mo em Caná, quando faltar em suas casas o amor sincero, oremos ao Senhor.
\end{paracol}

\textbf{P.} Senhor Jesus Cristo, que estais presente entre n\'os, agora que estes nossos irm\~aos \BG \ selam a sua u\-ni\-\~ao com um juramento de amor, aceitai as nossas preces e enchei-nos do vosso Esp\'irito. V\'os, que viveis e reinais para sempre.\\
\textbf{R.} Amém.


\begin{center}
\bfseries
LITURGIA EUCAR\'ISTICA
\end{center}

{\itshape \color{mygray}Tem início o Ofertório. Ao fim, todos colocam-se de p\'e e o sacerdote faz a oração sobre as oblatas.}

%\columnratio{0.5}
%\begin{paracol}{2}
%	\small
%	\itshape
%	\definecolor{mygray}{gray}{0.4}
%	\color{mygray}
%	\setlength{\parindent}{0em}
%	
%\textbf{Panis angelicus}\\
%(São Tomás de Aquino / \\ César Franck)
%
%\vspace{-0.2cm}
%Panis angelicus \\ fit panis hominum;\\
%Dat panis coelicus \\ figuris terminum
%
%\vspace{-0.2cm}
%O res mirabilis!\\
%Manducat Dominum\\
%Pauper, servus et humilis.
%
%\switchcolumn
%
%\textbf{Pão dos Anjos}\\
%
%\vspace{-0.7cm}
%O Pão dos Anjos \\ torna-se o pão dos homens,\\
%O Pão dos céus \\ dá fim às prefigurações.\\
%
%\vspace{-0.7cm}
%Ó coisa admirável,\\
%Alimentam-se do Senhor\\
%O pobre, o servo e os humildes.
%\end{paracol}

\textbf{P.} Acolhei, \'o Deus, as nossas s\'uplicas e oferendas por estes vossos filhos unidos pela alian\c ca do Matrim\^onio; que este sacramento os confirme em sua m\'utua caridade como tamb\'em no vosso amor. Por Cristo, nosso Senhor.\\
{\textbf{R.} Am\'em.}

\hfill {\itshape \color{mygray} Prefácio: Matrim\^onio, sinal do amor de Deus}

{\itshape \color{mygray}V.} O Senhor esteja convosco.\\
{\itshape \color{mygray}R.} Ele está no meio de nós.\\
{\itshape \color{mygray}V.} Corações ao alto.\\
{\itshape \color{mygray}R.} O nosso coração está em Deus.\\
{\itshape \color{mygray}V.} Demos graças ao Senhor nosso Deus.\\
{\itshape \color{mygray}R.} É nosso dever e nossa salvação.

\textbf{P.} Na verdade, \'e justo e necess\'ario, \'e nosso dever e salva\c c\~ao dar-vos gra\c cas sempre e em todo o lugar, Senhor, Pai santo, Deus eterno e todo-poderoso, por Cristo, Senhor nosso. A uni\~ao do homem e da mulher, que celebramos no sacramento do Matrim\^onio, \'e imagem de vosso Pai. De fato, por amor criastes o homem e a mulher e, na vossa bondade, os elevastes \`a dignidade de filha e filho vossos. Sempre de novo nos lembrais o mandamento do amor, distintivo de todos os vossos filhos e filhas. Sinal permanente do vosso amor \'e o sacramento do Matrim\^onio, que santifica o homem e a mulher para que possam participar da vossa eterna caridade. Unidos aos anjos e a todos os santos, n\'os vos aclamamos {\itshape dizendo} a uma s\'o voz:

\columnratio{0.15}
\begin{paracol}{2}
{\color{white}Blank.}

\switchcolumn

\itshape
Santo, Santo, Santo, Senhor Deus do Universo.\\
O céu e a terra proclamam a vossa glória.\\
Hosana nas alturas.\\
Bendito o que vem em nome do Senhor.\\
Hosana nas alturas.
\end{paracol}

%\pagebreak
\hfill {\itshape \color{mygray} Oração Eucarística II}

\textbf{P.} Na verdade, \'o Pai, v\'os sois santo e fonte de toda santidade. Santificai, pois, estas oferendas, derramando sobre elas o vos\-so Es\-p\'irito, a fim de que se tornem para n\'os o Corpo e {\itshape \color{mygray}\CrossMaltese} o Sangue de Jesus Cristo, vosso Filho e Senhor nosso.\\
\textbf{R.} Santificai nossa oferenda, \'o Senhor!

\textbf{P.} Estando para ser entregue e abra\c cando livremente a paix\~ao, ele tomou o p\~ao, deu gra\c cas, e o partiu e deu a seus disc\'ipulos, dizendo:

\columnratio{0.15}
\begin{paracol}{2}
{\color{white}Blank.}

\switchcolumn

\itshape
TOMAI, TODOS, E COMEI: ISTO É O MEU COR\-PO, QUE SERÁ ENTREGUE POR VÓS.
\end{paracol}

Do mesmo modo, ao fim da ceia, ele tomou o cálice em suas mãos, deu graças novamente, e o deu a seus discípulos, dizendo:

\columnratio{0.15}
\begin{paracol}{2}
{\color{white}Blank.}

\switchcolumn

\itshape
TOMAI, TODOS, E BEBEI: ESTE É O CÁLICE DO MEU SANGUE, O SANGUE DA NOVA E ETERNA ALIANÇA, QUE SERÁ DERRAMADO POR VÓS E POR TODOS PARA A REMISSÃO DOS PECADOS. FAZEI ISTO EM MEMÓRIA DE MIM.
\end{paracol}

\emph{Eis o mistério da fé!}\\
\textbf{R.} Anunciamos, Senhor, a vossa morte e proclamamos a vossa ressurreição. Vinde, Senhor Jesus!

\textbf{P.} Celebrando, pois, a mem\'oria da morte e ressurrei\c c\~ao do vosso Filho, n\'os vos oferecemos, \'o Pai, o p\~ao da vida e o c\'alice da salva\c c\~ao; e vos agradecemos porque nos tornastes dignos de estar aqui na vossa presen\c ca e vos servir.\\
\textbf{R.} Recebei, ó Senhor, a nossa oferta!

\textbf{P.} E n\'os vos suplicamos que, participando do Corpo e Sangue de Cristo, sejamos reunidos pelo Esp\'irito Santo num s\'o corpo.
\textbf{R.} Fazei de n\'os um s\'o corpo e um s\'o esp\'irito!

\textbf{P.} Lembrai-vos, \'o Pai, da vossa Igreja que se faz presente pelo mundo inteiro: que ela cres\c ca na caridade com o Papa \emph{N.}, com nosso Bispo \emph{N.} e todos os ministros do vosso povo. Lembrai-vos tamb\'em, Senhor, destes vossos filhos \BG. Assim como lhes destes a alegria do casamento, possam, por vossa gra\c ca, viver unidos no amor e na paz.\\
\textbf{R.} Lembrai-vos, \'o Pai, da vossa Igreja!

\textbf{P.} Lembrai-vos tamb\'em dos nossos irm\~aos e irm\~as que morreram na esperan\c ca da ressurrei\c c\~ao e de todos os que partiram desta vida: acolhei-os junto a v\'os na luz da vossa face.\\
\textbf{R.} Lembrai-vos, \'o Pai, dos vossos filhos!

\textbf{P.} Enfim, n\'os vos pedimos, tende piedade de todos n\'os e dai-nos participar da vida eterna, com a Virgem Maria, M\~ae de Deus, com s\~ao Jos\'e, seu esposo, com os santos Ap\'ostolos e todos os que neste mundo vos serviram, a fim de vos louvarmos e glorificarmos por Jesus Cristo, vosso Filho.\\
\textbf{R.} Concedei-nos o conv\'ivio dos eleitos!

\columnratio{0.15}
\begin{paracol}{2}
	{\color{white}Blank.}
	
	\switchcolumn
	
	\itshape
Por Cristo, com Cristo, em Cristo, a vós, Deus Pai todo-poderoso, na unidade do Espírito Santo, toda a honra e toda a glória, agora e para sempre.
\end{paracol}

\textbf{R.} \emph{Amém!}

%\vspace{-0.2cm}
%\pagebreak
\begin{center}
BÊNÇÃO NUPCIAL
\end{center}
\vspace{-0.2cm}

{\itshape \color{mygray}Ap\'os o {\color{black}Pai-nosso} e omitido o {\color{black}Livrai-nos de todo os males}, o sacerdote, de m\~aos postas e voltado para o casal, convida os fi\'eis \`a ora\c c\~ao:}

%\textbf{P.} Irm\~aos e irm\~as em Cristo, invoquemos as b\^en\c c\~aos divinas sobre este casal, para que Deus sustente com seu aux\'ilio aos que enriqueceu com o Matrim\^onio.

\textbf{P.} Caros fi\'eis, roguemos a Deus que derrame suas b\^en\c c\~aos sobre \BG, que se uniram em Cristo, pela alian\c ca sagrada do Matrim\^onio, para que se tornem um s\'o cora\c c\~ao pela caridade e pelo sacramento do Corpo e do Sangue de Cristo.

{\itshape \color{mygray}Todos oram em silêncio, durante alguns momentos.}

\textbf{P.} \'O Deus, santificastes misteriosamente a uni\~ao conjugal, des\-de o princ\'ipio, a fim de prefigurar no v\'inculo nupcial o mist\'erio do Cristo e da Igreja.

Volvei o vosso olhar de bondade sobre estes vossos filhos, que, unidos pelo v\'inculo do Matrim\^onio, esperam ser fortalecidos pela sua b\^en\c c\~ao: enviai sobre eles a gra\c ca do Esp\'irito Santo, para que, impregnados da vossa caridade, permane\c cam fi\'eis na alian\c ca conjugal.

O amor e a paz permane\c cam no cora\c c\~ao da vossa filha \textbf{B\'arbara}; e ela busque o exemplo das santas mulheres, exaltadas com louvores nas Sagradas Escrituras.

Nela confie o seu marido; e saiba honr\'a-la com a devida estima, reconhecendo-a companheira e co-herdeira da vida divina, e amando-a com aquele amor com que Cristo amou a sua Igreja.

N\'os vos pedimos, \'o Pai, que estes vossos filhos permane\c cam firmes na f\'e e amem os vossos mandamentos; que se conservem fi\'eis um ao outro e sejam para todos um exemplo.

Animados pela for\c ca do Evangelho, sejam entre todos verdadeiras testemunhas de Cristo. Sejam eles fecundos em filhos, pais de comprovada virtude, e possam ver os filhos de seus filhos.

Enfim, ap\'os uma vida longa e feliz, alcancem o reino do c\'eu e o conv\'ivio dos santos. Por Cristo, nosso Senhor.

%\textbf{P.} Deus, Pai Santo, que pelo vosso infinito poder fizestes do nada todas as coisas e, na harmonia primordial do universo, formastes o homem e a mulher à vossa imagem e semelhança, dando um ao outro como companheiros inseparáveis, para se tornarem os dois uma só carne, e assim nos ensinastes que nunca é lícito separar o que Vós mesmo unistes;
%
%Deus, Pai Santo, que no grande mistério do vosso amor consagrastes a aliança matrimonial, tornando-a símbolo da aliança de Cristo com a Igreja;
%
%Deus, Pai Santo, que sois o autor do matrimônio e destes à primordial comunidade humana a vossa bênção que nem a pena do pecado original nem o castigo do dilúvio nem criatura alguma pôde abolir;
%
%Olhai benignamente para estes vossos servos, que, unindo-se pelo vínculo do Matrimônio, esperam o auxílio da vossa bênção: enviai sobre eles a graça do Espírito Santo para que, pelo vosso amor derramado em seus corações, permaneçam fiéis na aliança conjugal.
%
%Seja a vossa serva \textbf{Bárbara} fortalecida com a graça do amor e da paz, imitando as santas mulheres que a Escritura tanto exalta.
%
%Confie nela o coração do seu marido, honrando-a como companheira igual em dignidade e com ele herdeira do dom da vida, e ame-a como Cristo amou a sua Igreja.
%
%Nós vos pedimos, Senhor, que estes vossos servos \BG \ permaneçam unidos na fé e na observância dos mandamentos; fiéis um ao outro, sirvam de exemplo pela integridade da sua vida; fortalecidos pela sabedoria do Evangelho, dêem a todos bom testemunho de Cristo; recebam o dom dos filhos, sejam pais de virtude comprovada, e possam ver os filhos dos seus filhos, e, depois de uma vida longa e feliz, alcancem o reino celeste, na companhia dos santos. Por Nosso Senhor Jesus Cristo vosso Filho, que é Deus convosco na unidade do Espírito Santo.

\textbf{R.} Amém.

{\itshape \color{mygray}Omitida a oração {\color{black}Senhor Jesus Cristo}, o sacerdote dir\'a imediatamente {\color{black}A paz do Senhor esteja sempre convosco}.}

%\pagebreak
\begin{center}
COMUNH\~AO EUCAR\'ISTICA
\end{center}

\hfill {\itshape \color{mygray}Oração depois da Comunh\~ao}

\textbf{P.} Concedei, \'o Deus todo-poderoso, que a gra\c ca do Matrim\^onio atinja a plenitude neste casal, e possamos todos colher os frutos do sacrif\'icio que oferecemos. Por Cristo, nosso Senhor.\\
\textbf{R.} Am\'em.

\begin{center}
BÊNÇÃO FINAL
\end{center}

\textbf{P.} Que Deus, vosso Pai, vos conserve no vosso amor, para que a paz de Cristo habite em v\'os e permane\c ca sempre na vossa casa.\\
\textbf{R.} Am\'em.

\textbf{P.} Que Deus vos d\^e a b\^en\c c\~ao dos filhos, o apoio dos amigos e a paz com todos.\\
\textbf{R.} Am\'em.

\textbf{P.} Sede no mundo um sinal do amor de Deus, abri vossa porta aos pobres e infelizes, que um dia vos receber\~ao, agradecidos, na casa do Pai.\\
\textbf{R.} Am\'em.

\textbf{P.} E a vós todos, aqui reunidos, abençoe-vos Deus todo-poderoso, Pai e Filho  {\color{mygray}\CrossMaltese} e Espírito Santo.\\
\textbf{R.} Amém.

\vfill
\pagebreak

\begin{center}
CANTOS EM LATIM E INGL\^ES
\end{center}
\columnratio{0.5}
\begin{paracol}{2}
\small
\definecolor{mygray}{gray}{0.4}
%\color{mygray}
\setlength{\parindent}{0em}
{\itshape
\textbf{With grateful hearts}\\
\textbf{(Hymn to St. Joseph)}\\
(P. J. Kennedy and Sons)


With grateful hearts we breathe today\\
The tender accents of our love\\
We carol forth a little lay\\
To thee, great Saint in heaven\\
above

O Joseph dear, from thy bright throne,\\
Incline thine ear unto our prayer\\
And over us all, as over thine own,\\
Extend thy fond paternal care

More favoured than earth's greatest king,\\
Thou wert the guardian of that Child,\\
Around whose crib full choirs did sing\\
With cadenced voices soft and mild
}
\switchcolumn

\textbf{Com corações agradecidos}\\
\textbf{(Hino a São José)}\\


Com corações agradecidos,\\ suspiramos hoje\\
doces expressões de nosso amor\\
Cantamos uma breve canção\\
para ti, grande Santo no céu


Ó querido José, do teu trono brilhante,\\
inclina o teu ouvido para a nossa\\ oração!\\
E sobre todos nós, como sendo teus próprios,\\
Derrama o teu afetuoso cuidado paternal


Mais favorecido que o maior rei da terra,\\
Tu foste o guardião daquela Criança,\\
Em torno de cujo berço coros numerosos cantavam\\
Com vozes cadenciadas, tenras e suaves\\
\end{paracol}
\vspace{-0.3cm}
%\begin{paracol}{2}
%\small
%\definecolor{mygray}{gray}{0.4}
%%	\color{mygray}
%\setlength{\parindent}{0em}
%{\itshape
%\textbf{You've got a friend in me}\\
%(Randy Newman)
%
%
%You've got a friend in me\\
%When the road looks rough ahead\\
%And you're miles and miles from your nice warm bed\\
%You just remember what your old pal said\\
%Boy, you've got a friend in me
%
%You've got a friend in me\\
%If you got troubles, I got'em too\\
%There isn't anything I wouldn't do for you\\
%If we stick together we can see it through\\
%'Cause you've got a friend in me\\
%Yes, you've got a friend in me
%}
%\switchcolumn
%
%\textbf{Você tem um amigo em mim}
%
%
%Você tem um amigo em mim\\
%Quando a estrada adiante parecer dura\\
%E você está há milhas e milhas de sua bela cama quente\\
%Você apenas relembre do que seu velho amigo disse\\
%Garoto, você tem um amigo em mim
%
%
%Você tem um amigo em mim\\
%Você tem seus problema, eu os tenho também\\
%Não há nada que eu não faria por você\\
%Nos mantendo unidos, nós podemos resolver isso\\
%Porque você tem um amigo em mim
%\end{paracol}


\noindent 
{\color{mygray}\hrule}

\columnratio{0.5}
\begin{paracol}{2}
\small
\definecolor{mygray}{gray}{0.4}
%	\color{mygray}
\setlength{\parindent}{0em}
{\itshape
\textbf{Panis angelicus}\\
(São Tomás de Aquino / \\ César Franck)

\vspace{-0.2cm}
Panis angelicus \\ fit panis hominum;\\
Dat panis coelicus \\ figuris terminum

\vspace{-0.2cm}
O res mirabilis!\\
Manducat Dominum\\
Pauper, servus et humilis.
}
\switchcolumn

\textbf{Pão dos Anjos}\\

\vspace{-0.7cm}
O Pão dos Anjos \\ torna-se o pão dos homens,\\
O Pão dos céus \\ dá fim às prefigurações.\\

\vspace{-0.7cm}
Ó coisa admirável,\\
Alimentam-se do Senhor\\
O pobre, o servo e os humildes.
\end{paracol}

\noindent 
{\color{mygray}\hrule}

\begin{paracol}{2}
\small
\definecolor{mygray}{gray}{0.4}
%	\color{mygray}
\setlength{\parindent}{0em}
{\itshape
\textbf{Can't help falling in love}\\
(Elvis Presley)\\

\vspace{-0.3cm}

Wise men say only fools rush in\\
But I can't help falling in love with you\\
Shall I stay?\\
Would it be a sin\\
If I can't help falling in love with you?\\
Like a river flows surely to the sea\\
Darling so it goes\\
Some things are meant to be\\

\vspace{-0.5cm}
Take my hand \\
Take my whole life too\\
For I can't help falling in love with you
}

\switchcolumn

\textbf{N\~ao consigo n\~ao me apaixonar por voc\^e}\\

\vspace{-0.4cm}
Homens sábios dizem\\
Que só os tolos se apaixonam\\
Mas eu não consigo evitar
me apaixonar por você\\
Eu deveria ficar?
Seria um pecado,\\
se eu não consigo evitar
me apaixonar por você?\\
Como um rio que corre\\
Certamente para o mar\\
Querida, é assim:
algumas coisas estão destinadas a acontecer \\

\vspace{-0.5cm}
Tome a minha m\~ao\\
Tome tamb\'em minha vida inteira\\
Pois n\~ao consigo n\~ao me apaixonar por voc\^e
\end{paracol}

\end{document}